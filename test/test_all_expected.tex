\documentclass[t]{beamer}
\definecolor{rrblitbackground}{rgb}{0.55, 0.3, 0.1}

\newenvironment{rtbliteral}{

\begin{ttfamily}

\color{rrblitbackground}

}{

\end{ttfamily}

}

\usetheme{Warsaw}

\setbeameroption{hide notes}

% generated by Docutils <http://docutils.sourceforge.net/>
\usepackage{fixltx2e} % LaTeX patches, \textsubscript
\usepackage{cmap} % fix search and cut-and-paste in Acrobat
\usepackage{ifthen}
\usepackage[T1]{fontenc}
\usepackage[latin1]{inputenc}
\setcounter{secnumdepth}{0}

%%% Custom LaTeX preamble
% PDF Standard Fonts
\usepackage{mathptmx} % Times
\usepackage[scaled=.90]{helvet}
\usepackage{courier}

%%% User specified packages and stylesheets

%%% Fallback definitions for Docutils-specific commands

% hyperlinks:
\ifthenelse{\isundefined{\hypersetup}}{
  \usepackage[colorlinks=true,linkcolor=blue,urlcolor=blue]{hyperref}
  \urlstyle{same} % normal text font (alternatives: tt, rm, sf)
}{}
\hypersetup{
  pdftitle={Test all features},
}


%%% Body
\begin{document}

% Document title
\title[Test all features]{Test all features%
  \label{test-all-features}}
\author[]{}
\date{}
\maketitle


\section{Introduction%
  \label{introduction}%
}

\begin{frame}[fragile]
\frametitle{Purpose}


The document is intended to test most features of rst2beamer in one go, such
any new versions can be quickly verified by calling:
\setbeamerfont{quote}{parent={}}
%
\begin{quote}{\ttfamily \raggedright \noindent
rst2beamer~test\_all.rst~test\_all.tex
}
\end{quote}
\setbeamerfont{quote}{parent=quotation}

However testing certain options will require running rst2beamer with flags.
These are marked below.

\end{frame}


\section{Testing sections%
  \label{testing-sections}%
}

\begin{frame}[fragile]
\frametitle{About testing sections}


If the sections feature works, this should be the first (and only) slide
\textquotedbl{}About testing sections\textquotedbl{} in the section \textquotedbl{}Testing sections\textquotedbl{}.

Note you may have to run LaTeX twice to get the section names to update
correctly.

\end{frame}


\section{Testing columns%
  \label{testing-columns}%
}

\begin{frame}[fragile]
\frametitle{Simple columns}

\begin{columns}[T]
\column{0.00\textwidth}

This is a demonstration of the rst2beamer simple column directive. It
should turn every element underneath it into a column, side by side with
each other.

\column{0.00\textwidth}

So here, we should end up with three columns of equal width, occupying 0.90
of the page in total (the default).

\column{0.00\textwidth}
\begin{itemize}[<+-| alert@+>]

\item Does it

\item handle lists

\item properly?
\end{itemize}

\end{columns}

\end{frame}

\begin{frame}[fragile]
\frametitle{Simple columns with a set width}

\begin{columns}[T]
\column{0.00\textwidth}

This is a demonstration of the rst2beamer simple column directive with a
set width.

\column{0.00\textwidth}

The total width has been set to 0.70.

\end{columns}

\end{frame}

\begin{frame}[fragile]
\frametitle{Testing containers as columns}

\begin{columns}[T]
\column{0.00\textwidth}

This uses a container to set out the columns.

\column{0.00\textwidth}

There should be two columns taking up 0.90 of the page.

\end{columns}

\end{frame}

\begin{frame}[fragile]
\frametitle{Testing explicit columns}

\begin{columns}[T]
\column{0.00\textwidth}

This tests the explicit column directive. No widths are given.

\column{0.00\textwidth}

There should be two columns sharing the default width of 0.90.

Note this column should have two paragraphs.

\end{columns}

\end{frame}

\begin{frame}[fragile]
\frametitle{Testing explicit columns with widths}

\begin{columns}[T]
\column{0.50\textwidth}

The column set has a width of 0.80, and this column 0.50.

\column{0.00\textwidth}

This column should get the remainder, 0.30.

\end{columns}

\end{frame}


\section{Testing notes%
  \label{testing-notes}%
}

\begin{frame}[fragile]
\frametitle{Introduction}


The notes on the follwing pages wil only show up if rst2beamer is run with
the \texttt{shownotes} option. For example:
\setbeamerfont{quote}{parent={}}
%
\begin{quote}{\ttfamily \raggedright \noindent
rst2beamer~-{}-shownotes~true~test\_all.rst~test\_all.tex
}
\end{quote}
\setbeamerfont{quote}{parent=quotation}

\end{frame}

\begin{frame}[fragile]
\frametitle{Testing the note directive}


There is a note on this page.
\note{

This is it.
}

\end{frame}

\begin{frame}[fragile]
\frametitle{Testing multiple note directives}


There are several notes on this page.
\note{

This is one.
}

Not that you should notice.
\note{

This is another.
}

Unless you use \textquotedbl{}shownotes\textquotedbl{}.
\note{

This is a third.
}

\end{frame}

\begin{frame}[fragile]
\frametitle{Notes as containers}


Notes can also be containers.
\note{

This is a note.
}

This helps with compatibility.
\note{

This is a second.
}

\end{frame}


\section{Other features%
  \label{other-features}%
}

\begin{frame}[fragile]
\frametitle{Bulletpoint overlays}


Normally the below list should appear as an overlay (i.e. point-by-point). It
will appear as a single unit if instead you call:
\setbeamerfont{quote}{parent={}}
%
\begin{quote}{\ttfamily \raggedright \noindent
rst2beamer~-{}-overlaybullets~false~test\_all.rst~test\_all.tex
}
\end{quote}
\setbeamerfont{quote}{parent=quotation}
\begin{itemize}[<+-| alert@+>]

\item Item one

\item Item two
\end{itemize}

\end{frame}

\begin{frame}[fragile]
\frametitle{Preformatted}


The below should appear as indented Python code with a monospace font:
\setbeamerfont{quote}{parent={}}
%
\begin{quote}{\ttfamily \raggedright \noindent
for~i~in~xrange~(10):\\
~~~~~~~~print~\textquotedbl{}foo\textquotedbl{},~i
}
\end{quote}
\setbeamerfont{quote}{parent=quotation}

\end{frame}

\begin{frame}[fragile]
\frametitle{Parsed literals}


The below should appear as indented Python code with a monospace font, and
some keywords in italics:
\setbeamerfont{quote}{parent={}}
%
\begin{quote}{\ttfamily \raggedright \noindent
\emph{for}~i~in~\emph{xrange}~(10):\\
~~~~~~~~\emph{print}~\textquotedbl{}foo\textquotedbl{},~i
}
\end{quote}
\setbeamerfont{quote}{parent=quotation}

\end{frame}

\begin{frame}[fragile]
\frametitle{Codeblocks}


The below should appear as a simple literal blocks, or highlighted if you use pygments:
\setbeamerfont{quote}{parent={}}
%
\begin{quote}{\ttfamily \raggedright \noindent
def~myfunc~(arg1,~arg2='foo'):\\
~~~~~~~~global~baz\\
~~~~~~~~bar~=~unicode~(quux)\\
~~~~~~~~return~25
}
\end{quote}
\setbeamerfont{quote}{parent=quotation}

\end{frame}

\end{document}
